\chapter{Теоретическая часть}
\section{Формулы для вычисления величин $M_{max}$ , $M_{min}$, R, $\hat\mu$, $S^2$}
Минимальное значение выборки рассчитывается по формуле (\ref{eq:min}); максимальное -- (\ref{eq:max}). Размах выборки рассчитывается по формуле (\ref{eq:diff}); выборочное среднее -- (\ref{eq:mx}), исправленная выборочная дисперсия -- (\ref{eq:dx}).
\begin{equation}
	\label{eq:min}
	M_{\min} = X_{(1)}
\end{equation}
\begin{equation}
	\label{eq:max}
	M_{\min} = X_{(n)}
\end{equation}
\begin{equation}
	\label{eq:diff}
	R = M_{\max} - M_{\min}.
\end{equation}
\begin{equation}
	\label{eq:mx}
	\hat\mu(\vec X_n) = \frac 1n \sum_{i=1}^n X_i
\end{equation}
\begin{equation}
	\label{eq:dx}
	S^2(\vec X_n) = \frac 1{n - 1} \sum_{i=1}^n (X_i-\overline X_n)^2
\end{equation}

\section{Эмпирическая плотность и гистограмма}
Пусть $\vec x$ -- выборка из генеральной совокупности $X$. 

При большом объеме n (n > 50) этой выборки  значения $x_i$ группируют в интервальный статистический ряд. Для этого отрезок $J = [x_{(1)}, x_{(n)}]$ делят на $m$ равновеликих промежутков по формуле (\ref{eq:ji}):

\begin{equation}
	\label{eq:ji}
	J_i = [x_{(1)} + (i - 1) \cdot \Delta,\ x_{(1)} + i \cdot \Delta), i = \overline{1; m - 1}
\end{equation}
Последний промежуток определяется по формуле (\ref{eq:jm}):
\begin{equation}
	\label{eq:jm}
	J_{m} = [x_{(1)} + (m - 1) \cdot \Delta, x_{(n)}]
\end{equation}

Ширина каждого из таких промежутков определяется по формуле (\ref{eq:delta}).
\begin{equation}
	\label{eq:delta}
	\Delta = \frac{|J|}{m} = \frac{x_{(n)} - x_{(1)}}{m}
\end{equation}

Интервальным статистическим рядом называют таблицу \ref{table:row1}:

\begin{table}[ht!]
	\captionsetup{singlelinecheck = false, justification=raggedleft}
	\caption{Интервальный статистический ряд}
	\centering
	\label{table:row1}
	\begin{tabular}{|c|c|c|c|c|}
		\hline
		$J_1$ & ... & $J_i$ & ... & $J_m$ \\
		\hline
		$n_1$ & ... & $n_i$ & ... & $n_m$ \\
		\hline
	\end{tabular}
\end{table}

где $n_i$ -- количество элементов выборки $\vec x$, которые $\in J_i$.

\textbf{Гистограмма} -- это график эмпирической плотности. 

\textbf{Эмпирической плотностью}, отвечающей выборке $\vec x$, называют функцию:
\begin{equation}
	\hat f(x) =
	\begin{cases}
		\frac{n_i}{n \Delta}, x \in J_i, i = \overline{1; m} \\
		0\ \ , x \not\in J \\
	\end{cases}
\end{equation}

где $J_i$ -- полуинтервал статистического ряда, 
$n_i$ -- количество элементов выборки, входящих в полуинтервал, 
$n$ -- количество элементов выборки.


\section{Эмпирическая функция распределения}

Пусть $\vec x = (x_1, ..., x_n)$ -- выборка из генеральной совокупности $X$. 

Обозначим $n(t, \vec x)$ -- число элементов вектора $\vec x$, которые имеют значения меньше $t$.

\textbf{Эмпирической функцией распределения} называют функцию \newline
$F_n: \mathbb{R} \to \mathbb{R}$, определенную как: 

\begin{equation}
	F_n(t) = \frac{n(t, \vec x)}{n}
\end{equation}